\documentclass[11pt]{article}
\bibliographystyle{siam}

\title{The title of your project proposal}
\author{
  LastName1, FirstName1\\
  \texttt{github1}
  \and
  LastName2, FirstName2\\
  \texttt{github2}
  \and
  LastName3, FirstName3\\
  \texttt{github3}
  \and
  LastName4, FirstName4\\
  \texttt{github4}
}

\begin{document}
\maketitle

Identify a published fMRI paper and the accompanying data
\cite{lindquist2008statistical}.  You should explain the basic idea of the
paper in a paragraph.  You should also perform basic sanity check on the data
(e.g., can you downloaded, can you load the files, confirm that you have the
correct number of subjects).



PART I: Data preprocess:
1. data clean up to remove outlier of the experiment if there is any.
For instance: use standard deviation to choose stable fMRI results.
2. collect and separate datasets based on subjects and object categories

PART II: statistic analysis of the dataset:
1. study wether the fMRI pattern for certain object category is consistent 
within individual subjects and across different subjects
2. using student t test to study whether the fMRI pattern for different 
categories are significant different between each other within individual 
subjects and across different subjects
3. study which object category has the most consistent fMRI pattern across 
different subjects. Which one has the least consistency. 
Create a ranking for consistency and see if is can be explained.

PART III: brain structure/cross talk study:
1. Study whether there is a specific brain region important for common 
object recognition. Find a region that is always activated when 
performing all experiment.
2. study whether there is a specific brain region for specific object 
recognition. For instance, is there a specific region for face recognition?
3. Study whether the activated brain regions are associated with brain 
regions that are known to mediate vision and/or cognition and/or memory.

PART IV: Machine learning and object prediction:
1. using machine learning approach to develop a model to predict observing 
object based on these fMRI results:
We will separate the datasets into mainly two groups. One of them will 
be used to develop machine learning models to predict observing objects. 
Then, we will use the other group to examine the accuracy of our 
prediction model.
a. using logistic regression to perform the machine learning
b. using neural network approach to perform the machine learning
2. Study if there is a specific category which gives us higher accuracy. 
For instance, face recognition is important for daily life. 
Does our brain recognize face with higher accuracy compare to others?
3. acquire new datasets from other sources to examine the accuracy of 
our model for predicting object recognition. (Ideally)

 

\bibliography{proposal}

\end{document}
